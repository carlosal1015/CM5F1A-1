\begin{frame}
	\begin{theorem}[El Principio de Inducción Matemática]
		Sea $F$ un \alert{cuerpo ordenado}.
		Suponga que $\forall n\in\mathbb{N}_{F}$, $p\left(n\right)$ es
		una proposición acerca de $n$.
		Si

		\begin{multicols}{2}
			\begin{enumerate}[(1)]
				\item\label{hyp:1}

				$p\left(1\right)$ es verdadero, y

				\item\label{hyp:2}

				$\forall k\in\mathbb{N}_{F}$,
				$p\left(k\right)\implies p\left(k+1\right)$,
			\end{enumerate}
		\end{multicols}

		entonces $\forall n\in\mathbb{N}_{F}$, $p\left(n\right)$ es
		verdadero.
	\end{theorem}

	\begin{proof}
		Suponga que $p\left(n\right)$ es como se describe en la
		hipótesis.
		Sea
		\begin{math}
			A=
			\left\{
			x\in\mathbb{N}_{F}:
			p\left(x\right)\text{ es verdadero}
			\right\}
		\end{math}.
		Entonces,
		\begin{enumerate}[(i)]
			\item

			      $1\in A$, por~\eqref{hyp:1}.

			\item

			      Suponga que $x\in A$. Entonces, $x\in\mathbb{N}_{F}$ y
			      $p\left(x\right)$ es verdadero.
			      Así, por~\eqref{hyp:2}, $p\left(x+1\right)$ es verdadero.
			      Esto es, $x+1\in A$.
			      Por lo tanto, $x\in A\implies x+1\in A$.
			      Finalmente, $A$ es conjunto inductivo y
			      $\mathbb{N}_{F}\subset A$.
			      Esto es, $\forall n\in\mathbb{N}_{F}$, $p\left(n\right)$
			      es verdadero.
		\end{enumerate}
	\end{proof}
\end{frame}