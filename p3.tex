\section{Pregunta N$^{\circ}$3\qquad Leon Alonzo Terrones Caccha}

\begin{frame}
    \begin{enumerate}\setcounter{enumi}{2}
        \item
              Demostrar que las funciones base de Bernstein para
              $n=3$ son linealmente independientes.
              Por consiguiente, son una base de $\mathbb{P}_{3}$
              (\alert{el espacio de polinomios de grado menor o igual
                  que $3$}).
              ¿Por qué basta con probar la independencia lineal?
              ¿Cómo se podría generalizar este argumento?
    \end{enumerate}

    \begin{solution}
        Sean
        \begin{math}
            {\left\{c_{k}\right\}}^{3}_{k=0}\subset\mathbb{R}
        \end{math}
        y
        \begin{math}
            {\left\{B_{k,3}\right\}}^{3}_{k=0}\subset\mathbb{P}_{3}
        \end{math}.
        Por demostrar que
        \begin{math}
            \sum\limits_{k=0}^{3}
            c_{k}
            B_{k,3}\left(t\right)=
            0\implies
            \forall k\in\left\{0,\dotsc,3\right\}:
            c_{k}=0
        \end{math}.

        \begin{align}
            \sum_{k=0}^{3}
            c_{k}
            B_{k,3}\left(t\right)
             & =
            \sum_{k=0}^{3}
            c_{k}
            \binom{3}{k}
            t^{k}
            \alert{
                {\left(1-t\right)}^{3-k}
            }=
            \sum_{k=0}^{3}
            c_{k}\binom{3}{k}t^k
            \alert{
                \sum_{j=0}^{3-k}
                \binom{3-k}{j}
                1^{j}
                    {\left(-t\right)}^{3-k-j}
            }\notag \\
            \label{eq:basis_bernstein}
            \Aboxed{
                \sum_{k=0}^{3}
                c_{k}
                B_{k,3}\left(t\right)
             & =
                \sum_{j=0}^{3}
                t^{n-j}
                \sum_{k=0}^{3-j}
                c_{k}
                \binom{3}{k}
                \binom{3-k}{j}
                \left(-1\right)^{3-k-j}\in\mathbb{P}_{3}.
            }       \\
            \label{eq:basis_standard}
            \Aboxed{
                \sum_{k=0}^{3}
                c_{k}
                B_{k,3}\left(t\right)
             & =
                \sum_{k=0}^{3}
                a_{k}
                t^{k}\in\mathbb{P}_{3}.
            }
        \end{align}
        Evaluamos para $j\in\left\{0,\dotsc,3\right\}$
        en~\eqref{eq:basis_bernstein} e igualamos
        con~\eqref{eq:basis_standard}.

        \begin{equation*}
            \begin{aligned}
                a_{0} & =
                \sum_{k=0}^{0}
                c_{k}
                \binom{3}{k}
                \binom{3-k}{3}
                {\left(-1\right)}^{-k}=
                c_{0}.    \\
                a_{1} & =
                \sum_{k=0}^{1}
                c_{k}
                \binom{3}{k}
                \binom{3-k}{2}
                {\left(-1\right)}^{1-k}=
                -3c_{0}+3c_{1}.
            \end{aligned}
            \qquad
            \begin{aligned}
                a_{2} & =
                \sum_{k=0}^{2}
                c_{k}
                \binom{3}{k}
                \binom{3-k}{1}
                {\left(-1\right)}^{2-k}=
                3c_{0}+-6c_{1}+3c_{2}. \\
                a_{3} & =
                \sum_{k=0}^{3}
                c_{k}
                \binom{3}{k}
                \binom{3-k}{0}
                {\left(-1\right)}^{3-k}=
                -c_{0}+3c_{1}-3c_{2}+c_{3}.
            \end{aligned}
        \end{equation*}
    \end{solution}
\end{frame}

\begin{frame}
    Así, se obtiene un sistema lineal
    \begin{equation*}
        \begin{bmatrix}
            c_{0} \\
            c_{1} \\
            c_{2} \\
            c_{3}
        \end{bmatrix}
    \end{equation*}
    Ya que
    \begin{math}
        \left\{
        1,
        t,
        t^{2},
        t^{3}
        \right\}
    \end{math}
    es una base para los polinomios de grado menor o igual que
    $3$ se tiene que
    \begin{math}
        a_{0}=
        a_{1}=
        a_{2}=
        a_{3}=
        0
    \end{math}.
    Resolviendo el sistema por sustitución progresiva se obtiene
    \begin{math}
        \forall k\in\left\{0,\dotsc,3\right\}:
        c_{k}=0
    \end{math}.
\end{frame}