\section{Pregunta N$^{\circ}$3\qquad Leon Alonzo Terrones Caccha}

\begin{frame}
    \begin{enumerate}\setcounter{enumi}{2}
        \item
              Demostrar que las funciones base de Bernstein para
              $n=3$ son linealmente independientes.
              Por consiguiente, son una base del espacio de
              polinomios de grado menor o igual que $3$.
              ¿Por qué basta con probar la independencia lineal?
              ¿Cómo se podría generalizar este argumento?
    \end{enumerate}

    \begin{solution}
        Sean $c_{0},c_{1},c_{2}\in\mathbb{R}$.
        Por demostrar que
        \begin{math}
            \sum\limits_{k=0}^{3}
            c_{k}
            B_{k,3}\left(t\right)=
            0\implies
            c_{0}=0,
            c_{1}=0,
            c_{2}=0
        \end{math}.

        \begin{align*}
            \sum_{k=0}^{3}
            c_{k}
            B_{k,3}\left(t\right)
             & =
            \sum_{k=0}^{3}
            c_{k}
            \binom{3}{k}
            t^{k}
            \alert{
                {\left(1-t\right)}^{3-k}
            }
            =
            \sum_{k=0}^{3}
            c_{k}\binom{3}{k}t^k
            \alert{
                \sum_{j=0}^{3-k}
                \binom{3-k}{j}
                1^{j}
                    {\left(-t\right)}^{3-k-j}
            }.
            \\
             & =
            \sum_{j=0}^{3}
            t^{n-j}
            \sum_{k=0}^{3-j}
            c_{k}
            \binom{3}{k}
            \binom{3-k}{j}
            \left(-1\right)^{3-k-j}
        \end{align*}
        es un polinomio de grado $3$ con coeficientes $a_{0}$,
        $a_{1}$ y $a_{2}$.
        Para $j=0$, el coeficiente de $t^{3}$ es

        \begin{align*}
            \sum_{k=0}^{3}{c_k\binom{3}{k}\binom{3-k}{0}(-1)^{3-k}} & =3(-3c_0+3c_1-3c_2+3c_3)
            a_{3}                                                   & =\sum_{k=0}^{3}{c_k\binom{3}{k}\binom{3-k}{0}(-1)^{3-k}} \\
                                                                    & =-c_0+3c_1-3c_2+c_3                                      \\
            a_{2}                                                   & =\sum_{k=0}^{2}{c_k\binom{3}{k}\binom{3-k}{1}(-1)^{2-k}} \\
                                                                    & =3c_0-6c_1+3c_2                                          \\
            a_{1}                                                   & =\sum_{k=0}^{2}{c_k\binom{3}{k}\binom{3-k}{1}(-1)^{2-k}} \\
                                                                    & =-3c_0+3c_1                                              \\
            a_{0}                                                   & =\sum_{k=0}^{2}{c_k\binom{3}{k}\binom{3-k}{1}(-1)^{2-k}} \\
                                                                    & =c_0
        \end{align*}
    \end{solution}
\end{frame}

\begin{frame}
    \begin{solution}
        El cual es un polinomio de grado $3$.
        Entonces sean $a_{j}$ los coeficientes de $t^{j}$.
        Se tiene:
        \begin{align*}
            a_3 & =\sum_{k=0}^{3}{c_k\binom{3}{k}\binom{3-k}{0}(-1)^{3-k}}=-c_0+3c_1-3c_2+c_3 \\
            a_2 & =\sum_{k=0}^{2}{c_k\binom{3}{k}\binom{3-k}{1}(-1)^{2-k}}=3c_0+-6c_1+3c_2    \\
            a_1 & =\sum_{k=0}^{1}{c_k\binom{3}{k}\binom{3-k}{2}(-1)^{1-k}}=-3c_0+3c_1         \\
            a_0 & =\sum_{k=0}^{0}{c_k\binom{3}{k}\binom{3-k}{3}(-1)^{-k}}=c_0                 \\
        \end{align*}

        Al final se obtiene un sistema lineal en función del vector $c=(c_0,c_1,c_2,c_3)$. Debido a que $\{1,t,t^2,t^3\}$ es una base para los polinomios de grado 3 se tiene que $a_k=0$. Finalmente aplicando sustitución progresiva se obtiene $c_k=0$, $k\in[0.3]$.
    \end{solution}
\end{frame}
