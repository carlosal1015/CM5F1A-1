\section{Pregunta N$^{\circ}$3\qquad Leon Alonzo Terrones Caccha}

\begin{frame}
    \begin{enumerate}\setcounter{enumi}{2}
        \item
              Demostrar que las funciones base de Bernstein para
              $n=3$ son linealmente independientes.
              Por consiguiente, son una base del espacio de
              polinomios de grado menor o igual que $3$.
              ¿Por qué basta con probar la independencia lineal?
              ¿Cómo se podría generalizar este argumento?
    \end{enumerate}

    \begin{solution}
        Sean $c_{0},c_{1},c_{2},c_{3}\in\mathbb{R}$.
        Por demostrar que
        \begin{math}
            \sum\limits_{k=0}^{3}
            c_{k}
            B_{k,3}\left(t\right)=
            0\implies
            c_{0}=
            c_{1}=
            c_{2}=
            c_{3}=0
        \end{math}.

        \begin{equation*}
            \sum_{k=0}^{3}
            c_{k}
            B_{k,3}\left(t\right)=
            \sum_{k=0}^{3}
            c_{k}
            \binom{3}{k}
            t^{k}
            \alert{
                {\left(1-t\right)}^{3-k}
            }=
            \sum_{k=0}^{3}
            c_{k}\binom{3}{k}t^k
            \alert{
                \sum_{j=0}^{3-k}
                \binom{3-k}{j}
                1^{j}
                    {\left(-t\right)}^{3-k-j}
            }=
            \sum_{j=0}^{3}
            t^{n-j}
            \sum_{k=0}^{3-j}
            c_{k}
            \binom{3}{k}
            \binom{3-k}{j}
            \left(-1\right)^{3-k-j}
        \end{equation*}
        es un polinomio de grado $3$.
        Sean $a_{j}$ los coeficientes de $t^{j}$.
        \begin{equation*}
            \begin{aligned}
                a_{3} & =
                \sum_{k=0}^{3}
                c_{k}
                \binom{3}{k}
                \binom{3-k}{0}
                {\left(-1\right)}^{3-k}=
                -c_{0}+3c_{1}-3c_{2}+c_{3}. \\
                a_{2} & =
                \sum_{k=0}^{2}
                c_{k}
                \binom{3}{k}
                \binom{3-k}{1}
                {\left(-1\right)}^{2-k}=
                3c_{0}+-6c_{1}+3c_{2}.
            \end{aligned}
            \qquad
            \begin{aligned}
                a_{1} & =
                \sum_{k=0}^{1}
                c_{k}
                \binom{3}{k}
                \binom{3-k}{2}
                {\left(-1\right)}^{1-k}=
                -3c_{0}+3c_{1}. \\
                a_{0} & =
                \sum_{k=0}^{0}
                c_{k}
                \binom{3}{k}
                \binom{3-k}{3}
                {\left(-1\right)}^{-k}=
                c_{0}.
            \end{aligned}
        \end{equation*}
        Así, se obtiene un sistema lineal en función del vector
        \begin{math}
            c=
            \left(
            c_{0},
            c_{1},
            c_{2},
            c_{3}
            \right)
        \end{math}.
        Ya que
        \begin{math}
            \left\{
            1,
            t,
            t^{2},
            t^{3}
            \right\}
        \end{math}
        es una base para los polinomios de grado menor o igual que
        $3$ se tiene que
        \begin{math}
            a_{0}=
            a_{1}=
            a_{2}=
            a_{3}=
            0
        \end{math}.
        Finalmente, aplicando sustitución progresiva se obtiene
        \begin{math}
            c_{0}=
            c_{1}=
            c_{2}=
            c_{3}=
            0
        \end{math}.
    \end{solution}
\end{frame}