\section{Pregunta N$^{\circ}$3\qquad Leon Alonzo Terrones Caccha}

\begin{frame}
    \begin{enumerate}\setcounter{enumi}{2}
        \item
              Demostrar que las funciones base de Bernstein para
              $n=3$ son linealmente independientes.
              Por consiguiente, son una base de $\mathbb{P}_{3}$
              (\alert{el espacio de polinomios de grado menor o igual
                  que $3$}).
              ¿Por qué basta con probar la independencia lineal?
              ¿Cómo se podría generalizar este argumento?
    \end{enumerate}

    \begin{solution}
        Sean
        \begin{math}
            {\left\{c_{k}\right\}}^{3}_{k=0}\subset\mathbb{R}
        \end{math}
        y
        \begin{math}
            {\left\{B_{k,3}\right\}}^{3}_{k=0}\subset\mathbb{P}_{3}
        \end{math}.
        Por demostrar que
        \begin{math}
            \sum\limits_{k=0}^{3}
            c_{k}
            B_{k,3}\left(t\right)=
            0\implies
            \forall k\in\left\{0,\dotsc,3\right\}:
            c_{k}=0
        \end{math}.

        \begin{align}
            \sum_{k=0}^{3}
            c_{k}
            B_{k,3}\left(t\right)
             & =
            \sum_{k=0}^{3}
            c_{k}
            \binom{3}{k}
            t^{k}
            \alert{
                {\left(1-t\right)}^{3-k}
            }=
            \sum_{k=0}^{3}
            c_{k}\binom{3}{k}t^k
            \alert{
                \sum_{j=0}^{3-k}
                \binom{3-k}{j}
                1^{j}
                    {\left(-t\right)}^{3-k-j}
            }.\notag \\
            \label{eq:basis_bernstein}
            \Aboxed{
                \sum_{k=0}^{3}
                c_{k}
                B_{k,3}\left(t\right)
             & =
                \sum_{j=0}^{3}
                t^{3-j}
                \sum_{k=0}^{3-j}
                c_{k}
                \binom{3}{k}
                \binom{3-k}{j}
                \left(-1\right)^{3-k-j}\in\mathbb{P}_{3}.
            }        \\
            \label{eq:basis_standard}
            \Aboxed{
                \sum_{k=0}^{3}
                c_{k}
                B_{k,3}\left(t\right)
             & =
                \sum_{k=0}^{3}
                a_{k}
                t^{k}\in\mathbb{P}_{3}.
            }
        \end{align}
        Igualando los coeficientes del mismo grado
        en~\eqref{eq:basis_bernstein} y~\eqref{eq:basis_standard},
        es decir, para $j\in\left\{3,\dotsc,0\right\}$:

        \begin{equation*}
            \begin{aligned}
                a_{0} & =
                \sum_{k=0}^{3-\alert{3}}
                c_{k}
                \binom{3}{k}
                \binom{3-k}{\alert{3}}
                {\left(-1\right)}^{3-k-\alert{3}}=
                c_{0}.    \\
                a_{1} & =
                \sum_{k=0}^{3-\alert{2}}
                c_{k}
                \binom{3}{k}
                \binom{3-k}{\alert{2}}
                {\left(-1\right)}^{3-k-\alert{2}}=
                -3c_{0}+3c_{1}.
            \end{aligned}
            \qquad
            \begin{aligned}
                a_{2} & =
                \sum_{k=0}^{3-\alert{1}}
                c_{k}
                \binom{3}{k}
                \binom{3-k}{\alert{1}}
                {\left(-1\right)}^{3-k-\alert{1}}=
                3c_{0}+-6c_{1}+3c_{2}. \\
                a_{3} & =
                \sum_{k=0}^{3-\alert{0}}
                c_{k}
                \binom{3}{k}
                \binom{3-k}{\alert{0}}
                {\left(-1\right)}^{3-k-\alert{0}}=
                -c_{0}+3c_{1}-3c_{2}+c_{3}.
            \end{aligned}
        \end{equation*}
    \end{solution}
\end{frame}

\begin{frame}
    \begin{solution}
        Pero,
        \begin{math}
            \left\{
            1,
            t,
            t^{2},
            t^{3}
            \right\}\subset\mathbb{P}_{3}
        \end{math}
        es un conjunto linealmente independiente.
        O sea,
        \begin{math}
            \sum\limits_{k=0}^{3}
            a_{k}
            t^{k}=
            0\implies
            \forall k\in\left\{0,\dotsc,3\right\}:
            a_{k}=0
        \end{math}.

        Así, resolvemos el sistema lineal por sustitución progresiva
        \begin{equation*}
            \begin{bmatrix}
                1  & 0  & 0  & 0 \\
                -3 & 3  & 0  & 0 \\
                3  & -6 & 3  & 0 \\
                -1 & 3  & -3 & 1
            \end{bmatrix}
            \begin{bmatrix}
                c_{0} \\
                c_{1} \\
                c_{2} \\
                c_{3}
            \end{bmatrix}=
            \begin{bmatrix}
                a_{0} \\
                a_{1} \\
                a_{2} \\
                a_{3}
            \end{bmatrix}=
            \begin{bmatrix}
                0 \\
                0 \\
                0 \\
                0
            \end{bmatrix}.
        \end{equation*}

        \begin{math}
            \therefore
            \forall k\in\left\{0,\dotsc,3\right\}:
            c_{k}=0
        \end{math}.

        Basta con probar la independientes lineal de las funciones
        base de Bernstein para afirmar que es una base de
        $\mathbb{P}_3$ en virtud del siguiente teorema.

        \begin{theorem}[Teorema de la dimensión para espacios vectoriales]
            Sea $V$ un espacio vectorial que admite una base de $n$
            vectores.
            Entonces, cualquier sistema de vectores linealmente
            independiente de $V$ tiene a lo más $n$ elementos y
            cualquier otra base de $V$ tiene $n$ elementos.
        \end{theorem}

        Este teorema también es se generaliza para espacios vectoriales de
        dimensión infinita.

        \begin{theorem}
            Dado un espacio vectorial $V$, cualesquiera dos bases de
            $V$ tienen la misma cardinalidad.
        \end{theorem}
    \end{solution}
\end{frame}