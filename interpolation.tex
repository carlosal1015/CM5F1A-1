\begin{frame}
	\frametitle{Polinomio de interpolación}

	Sean $n+1$ puntos distintos
	\begin{math}
		{
			\left\{
			\left(x_{k},y_{k}\right)
			\right\}
		}_{k=0}^{n}\subset
		\left[a,b\right]\times\mathbb{R}
	\end{math}
	y
	\begin{math}
		f\colon\left[a,b\right]\to
		\mathbb{R}
	\end{math}
	una función de modo que
	\begin{math}
		y_{k}=
		f\left(x_{k}\right)
	\end{math}
	para $0\leq k\leq n$.

	\begin{definition}[Polinomio de interpolación en la forma de Lagrange]
		\begin{equation*}
			\Pi_{n}
			f\left(x\right)\coloneqq
			% y_{0}
			% \ell_{0}\left(x\right)+
			% y_{1}
			% \ell_{1}\left(x\right)+
			% \cdots+
			% y_{n}
			% \ell_{n}\left(x\right)=
			\sum\limits_{k=0}^{n}
			y_{k}
			\ell_{k}\left(x\right)\in\mathbb{P}_{n},
		\end{equation*}
		donde
		\begin{math}
			\ell_{k}
			\left(x\right)\coloneqq
			\prod\limits_{\substack{j=0\\j\neq k}}^{n}
			\dfrac{x-x_{j}}{x_{k}-x_{j}}
		\end{math}
		para $0\leq k\leq n$ son los \alert{polinomios característicos}
		que satisface
		\begin{math}
			\ell_{k}
			\left(x_{j}\right)=
			\delta_{kj}
		\end{math}.

		La evaluación de $\Pi_{n}f\left(x\right)$ requiere
		$O\left(n^{2}\right)$ sumas y productos, en general el algoritmo
		es \emph{numéricamente inestable}.
	\end{definition}
	% Here $\ell_{0},\ell_{1},\ldots,\ell_{n}$ are polynomials that
	% depend on the nodes $x_0, x_1, \ldots, x_n$ but not on the
	% ordinates $y_{0},y_{1},\ldots,y_{n}$.
	% Since all the ordinates could be $0$ except for a $1$ occupying the
	% $i$-th position, we see that

	% \begin{equation*}
	% 	\delta_{ij}=
	% 	p_{n}
	% 	\left(x_j\right)=
	% 	\sum\limits_{k=0}^{n}
	% 	y_{k}
	% 	\ell_k\left(x_j\right)=
	% 	\sum\limits_{k=0}^{n}
	% 	\delta_{ki}
	% 	\ell_{k}
	% 	\left(x_j\right)=
	% 	\ell_{i}
	% 	\left(x_j\right).
	% \end{equation*}

	% (Recall that the Kronecker delta is defined by $\delta_{k i}=1$ if
	% $k=i$ and $\delta_{k i}=0$ if $k \neq i$.)
	% We can easily arrive at a set of polynomials having this property.
	% Let us consider $\ell_{0}$.
	% It is to be a polynomial of degree $n$ that takes the value $0$ at
	% $x_{1},x_{2},\ldots,x_{n}$ and the value $1$ at $x_0$.
	% Clearly, $\ell_{0}$ must be of the form

	% \begin{equation*}
	% 	\ell_{0}
	% 	\left(x\right)=
	% 	c
	% 	\left(x-x_1\right)
	% 	\left(x-x_2\right)\cdots
	% 	\left(x-x_n\right)=
	% 	c
	% 	\prod\limits_{j=1}^{n}
	% 	\left(x-x_j\right).
	% \end{equation*}

	% The value of $c$ is obtained by putting $x=x_{0}$, so that
	% \begin{math}
	% 	1=
	% 	c
	% 	\prod\limits_{j=1}^{n}
	% 	\left(x_{0}-x_{j}\right)
	% \end{math}
	% y
	% \begin{math}
	% 	c=
	% 	\prod\limits_{j=1}^{n}
	% 	{\left(x_{0}-x_{j}\right)}^{-1}
	% \end{math}.

	\begin{definition}[Polinomio de interpolación en la forma de Newton]
		\begin{equation*}
			\Pi_{n}
			f\left(x\right)\coloneqq
			\sum\limits_{k=0}^{n}
			a_{k}
			\omega_{k}\left(x\right)\in\mathbb{P}_{n},
		\end{equation*}
		donde
		\begin{itemize}
			\item

			      \begin{math}
				      a_{k}\coloneqq
				      f\left[x_{0},\ldots,x_{k}\right]
			      \end{math}
			      es la \alert{$k$-ésima diferencia dividida de Newton}, y

			\item

			      \begin{math}
				      \omega_{k}
				      \left(x\right)\coloneqq
				      \prod\limits_{j=0}^{k-1}
				      \left(
				      x-x_{j}
				      \right)
			      \end{math}
			      es el \alert{polinomio nodal de grado $k$}.
		\end{itemize}
		La evaluación de $\Pi_{n}f\left(x\right)$ requiere
		$O\left(n\right)$.
	\end{definition}
\end{frame}

% https://people.maths.ox.ac.uk/trefethen/barycentric.pdf
\begin{frame}
	\begin{definition}[Interpolación baricéntrica de Lagrange]
		Con el fin de realizar menos operaciones en la interpolación
		polinomial de Lagrange, multiplicamos por
		\begin{math}
			\alert{
				\dfrac{1}{\omega_{n+1}\left(x\right)}
			}
		\end{math}
		y resulta
		\begin{align*}
			\alert{
				\dfrac{1}{\omega_{n+1}\left(x\right)}
			}
			\Pi_{n}f\left(x\right) & =
			\alert{
				\dfrac{1}{\omega_{n+1}\left(x\right)}
			}
			\sum\limits_{k=0}^{n}
			y_{k}
			\alert{
				\ell_{k}\left(x\right)
			}=
			\dfrac{1}{\omega_{n+1}\left(x\right)}
			\sum\limits_{k=0}^{n}
			y_{k}
			\alert{
			\prod\limits_{\substack{j=0 \\j\neq k}}^{n}
			\dfrac{x-x_{j}}{x_{k}-x_{j}}
			}.
			\\
			                       & =
			\sum\limits_{k=0}^{n}
			\left\{
			\dfrac{y_{k}}{
				\alert{
			\prod\limits_{\substack{j=0 \\j\neq k}}^{n}
					\left(
					x_{k}-x_{j}
					\right)
				}
			}
			\dfrac{
				\alert{
			\prod\limits_{\substack{j=0 \\j\neq k}}^{n}
					\left(
					x-x_{j}
					\right)
				}
			}{\omega_{n+1}\left(x\right)}
			\right\}=
			\sum\limits_{k=0}^{n}
			\left\{
			\dfrac{y_{k}}{
			\prod\limits_{\substack{j=0 \\j\neq k}}^{n}
				\left(
				x_{k}-x_{j}
				\right)
			}
			\dfrac{1}{x-x_{k}}
			\right\}.
			\\
			\Pi_{n}f\left(x\right)
			                       & =
			\sum\limits_{k=0}^{n}
			y_{k}
			\widetilde{\ell}_{k}\left(x\right)\in\mathbb{P}_{n},
		\end{align*}
		donde
		\begin{columns}
			\begin{column}{.45\paperwidth}
				\begin{itemize}
					\item

					      \begin{math}
						      \widetilde{\ell}_{k}
						      \left(x\right)=
						      \omega_{n+1}
						      \left(x\right)
						      \dfrac{b_{k}}{x-x_{k}}
					      \end{math}, y
				\end{itemize}
			\end{column}
			\begin{column}{.45\paperwidth}
				\begin{itemize}
					\item

					      \begin{math}
						      b_{j}=
						      \dfrac{1}{
							      \prod\limits_{\substack{j=0\\j\neq k}}^{n}
							      \left(
							      x_{k}-x_{j}
							      \right)
						      }
					      \end{math}
					      son los \emph{pesos baricéntricos}.
				\end{itemize}
			\end{column}
		\end{columns}
		% La evaluación de $\Pi_{n}f\left(x\right)$ requiere
		% $O\left(n\right)$.
	\end{definition}
\end{frame}