\begin{frame}
    Sea $\gamma\colon\left[a,b\right]\to\mathbb{R}^{n}$ una curva
    suave.
    % \begin{definition}[Longitud de arco]
    %     La función \alert{longitud de arco} es dada por
    %     \begin{math}
    %         \displaystyle
    %         s\left(t\right)=
    %         \int\limits_{a}^{t}
    %         \left\|
    %         \gamma^{\prime}\left(u\right)
    %         \right\|\dl u
    %     \end{math}.
    % \end{definition}

    \begin{definition}[Curvatura]
        La \alert{curvatura} es
        \begin{math}
            k\left(t\right)=
            \dfrac{
                \left\|
                T^{\prime}\left(t\right)
                \right\|
            }{
                \left\|
                \gamma^{\prime}\left(t\right)
                \right\|
            }
        \end{math},
        donde
        \begin{math}
            T\left(t\right)=
            \dfrac{
                \gamma^{\prime}\left(t\right)
            }{
                \left\|
                \gamma^{\prime}\left(t\right)
                \right\|
            }
        \end{math}.
    \end{definition}

    \begin{theorem}
        Si $n=3$, entonces
        \begin{math}
            k\left(t\right)=
            \dfrac{
                \left\|
                \gamma^{\prime}\left(t\right)\times
                \gamma^{\prime\prime}\left(t\right)
                \right\|
            }{
                {\left\|
                        \gamma^{\prime}\left(t\right)
                        \right\|}^{3}
            }
        \end{math}.
    \end{theorem}

    \begin{theorem}[Curvatura sin signo de la gráfica de una función]
        Si
        \begin{math}
            \gamma\left(t\right)=
            \left(
            t,
            f\left(t\right)
            \right)
        \end{math}
        es la gráfica de la función
        \begin{math}
            f\colon
            \left[a,b\right]\to
            \mathbb{R}
        \end{math}
        dos veces derivable en
        $\left(a,b\right)$, entonces la curvatura es
        \begin{equation*}
            \kappa\left(t\right)=
            \dfrac{
                \left|
                f^{\prime\prime}
                \left(t\right)
                \right|
            }{
                {\left(
                        1+
                        {\left(f^{\prime}\left(t\right)\right)}^{2}
                        \right)}^{\frac{3}{2}}
            }.
        \end{equation*}
    \end{theorem}

    \begin{proof}
        \begin{equation*}
            k\left(t\right)=
            \dfrac{
                \left\|
                \gamma^{\prime}\left(t\right)\times
                \gamma^{\prime\prime}\left(t\right)
                \right\|
            }{
                {\left\|
                        \gamma^{\prime}\left(t\right)
                        \right\|}^{3}
            }=
            \dfrac{
                \left\|
                \left(
                1,f^{\prime}\left(t\right)
                \right)\times
                \left(
                0,f^{\prime\prime}\left(t\right)
                \right)
                \right\|
            }{
                {\left\|
                        \left(
                        1,f^{\prime}\left(t\right)
                        \right)
                        \right\|}^{3}
            }=
            \dfrac{
                \left|
                f^{\prime\prime}
                \left(t\right)
                \right|
            }{
                {\left(
                        \sqrt{
                            {\left(1\right)}^{2}+
                            {\left(f^{\prime}\left(t\right)\right)}^{2}
                        }
                        \right)}^{3}
            }.
        \end{equation*}
    \end{proof}
\end{frame}
% https://openstax.org/books/calculus-volume-3/pages/3-3-arc-length-and-curvature#:~:text=The%20curvature%20of%20the%20graph,radius%20of%20the%20inscribed%20circle.