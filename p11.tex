\section{Pregunta N$^{\circ}$11\qquad Andre Gilmer Santos Felix}

\begin{frame}
	\begin{theorem}[El Principio de Inducción Matemática]
		Sea $F$ un \alert{cuerpo ordenado}.
		Suponga que $\forall n\in\mathbb{N}_{F}$, $p\left(n\right)$ es
		una proposición acerca de $n$.
		Si

		\begin{multicols}{2}
			\begin{enumerate}[(1)]
				\item\label{hyp:1}

				$p\left(1\right)$ es verdadero, y

				\item\label{hyp:2}

				$\forall k\in\mathbb{N}_{F}$,
				$p\left(k\right)\implies p\left(k+1\right)$,
			\end{enumerate}
		\end{multicols}

		entonces $\forall n\in\mathbb{N}_{F}$, $p\left(n\right)$ es
		verdadero.
	\end{theorem}

	\begin{proof}
		Suponga que $p\left(n\right)$ es como se describe en la
		hipótesis.
		Sea
		\begin{math}
			A=
			\left\{
			x\in\mathbb{N}_{F}:
			p\left(x\right)\text{ es verdadero}
			\right\}
		\end{math}.
		Entonces,
		\begin{enumerate}[(i)]
			\item

			      $1\in A$, por~\eqref{hyp:1}.

			\item

			      Suponga que $x\in A$. Entonces, $x\in\mathbb{N}_{F}$ y
			      $p\left(x\right)$ es verdadero.
			      Así, por~\eqref{hyp:2}, $p\left(x+1\right)$ es verdadero.
			      Esto es, $x+1\in A$.
			      Por lo tanto, $x\in A\implies x+1\in A$.
			      Finalmente, $A$ es conjunto inductivo y
			      $\mathbb{N}_{F}\subset A$.
			      Esto es, $\forall n\in\mathbb{N}_{F}$, $p\left(n\right)$
			      es verdadero.
		\end{enumerate}
	\end{proof}
\end{frame}

\begin{frame}
	\begin{enumerate}\setcounter{enumi}{10}
		\item

		      Pruebe que el
		      \begin{math}
			      \begin{vmatrix}
				      A_{n}
			      \end{vmatrix}=
			      \prod\limits_{\mathclap{0\leq i< j\leq n}}
			      \left(t_{j}-t_{i}\right)
		      \end{math},
		      donde
		      \begin{math}
			      A_{n}=
			      \begin{bmatrix}
				      1      & \cdots & t_{0}^{n} \\
				      \vdots & \ddots & \vdots    \\
				      1      & \cdots & t_{n}^{n}
			      \end{bmatrix}
		      \end{math}
		      es la matriz de Vandermonde.
	\end{enumerate}

	\begin{solution}
		Sean $n\in\mathbb{N}$ y
		\begin{math}
			p\left(n\right)\coloneqq
			\left|A_{n}\right|=
			\prod\limits_{\mathclap{0\leq i< j\leq n}}
			\left(t_{j}-t_{i}\right)
		\end{math}.
		Por el \alert{Principio de Inducción Matemática} sobre $n$,

		\begin{itemize}
			\item

			      \begin{math}
				      p\left(1\right)=
				      \begin{vmatrix}
					      A_{1}
				      \end{vmatrix}=
				      \begin{vmatrix}
					      1 & t_{0} \\
					      1 & t_{1}
				      \end{vmatrix}=
				      t_{1}-t_{0}=
				      \prod\limits_{\mathclap{0\leq i< j\leq 1}}
				      t_{j}-t_{i}
			      \end{math}
			      es verdadero.

			\item

			      Supongamos que $p\left(k\right)$ es verdadero,
			      es decir,
			      \begin{math}
				      \left|
				      A_{k}
				      \right|=
				      \prod\limits_{\mathclap{0\leq i< j\leq k}}
				      \left(t_{j}-t_{i}\right)=
				      \begin{vmatrix}
					      1      & \cdots & t_{0}^{k} \\
					      \vdots & \ddots & \vdots    \\
					      1      & \cdots & t_{k}^{k}
				      \end{vmatrix}.
			      \end{math}
			      Veamos que $p\left(k+1\right)$ es verdadero.

			      \begin{align*}
				      \left|
				      A_{k+1}
				      \right| & =
				      \begin{vmatrix}
					      1      & t_{0}   & \cdots & t_{0}^{k+1}   \\
					      1      & t_{1}   & \cdots & t_{1}^{k+1}   \\
					      \vdots & \vdots  & \ddots & \vdots        \\
					      1      & t_{k+1} & \cdots & t_{k+1}^{k+1}
				      \end{vmatrix}
				      \overset{\left(\star\right)}{=}
				      \begin{vmatrix}
					      1      & t_{0}-\alert{1}\cdot t_{0}   & \cdots & t_{0}^{k+1}-\alert{t_{0}^{k}}\cdot t_{0}     \\
					      1      & t_{1}-\alert{1}\cdot t_{0}   & \cdots & t_{1}^{k+1}-\alert{t_{1}^{k}}\cdot t_{0}     \\
					      \vdots & \vdots                       & \ddots & \vdots                                       \\
					      1      & t_{k+1}-\alert{1}\cdot t_{0} & \cdots & t_{k+1}^{k+1}-\alert{t_{k+1}^{k}}\cdot t_{0}
				      \end{vmatrix}=
				      \begin{vmatrix}
					      1      & 0             & \cdots & 0                                   \\
					      1      & t_{1}-t_{0}   & \cdots & \left(t_{1}-t_{0}\right)t_{1}^{k}   \\
					      \vdots & \vdots        & \ddots & \vdots                              \\
					      1      & t_{k+1}-t_{0} & \cdots & \left(t_{k+1}-t_{0}\right)t_{k}^{k}
				      \end{vmatrix}. \\
				              & =
				      \begin{vmatrix}
					      t_{1}-t_{0}   & \cdots & \left(t_{1}-t_{0}\right)t_{1}^{k}   \\
					      \vdots        & \ddots & \vdots                              \\
					      t_{k+1}-t_{0} & \cdots & \left(t_{k+1}-t_{0}\right)t_{k}^{k}
				      \end{vmatrix}=
				      \prod\limits_{\mathclap{0 <j\leq k+1}}
				      \left(t_{j}-t_{0}\right)
				      \alert{
					      \begin{vmatrix}
						      1      & \cdots & t_{1}^{k} \\
						      \vdots & \ddots & \vdots    \\
						      1      & \cdots & t_{k}^{k}
					      \end{vmatrix}
				      }\overset{\text{H.I.}}{=}
				      \prod\limits_{\mathclap{0 <j\leq k+1}}
				      \left(t_{j}-t_{0}\right)
				      \alert{
					      \prod\limits_{\mathclap{0\leq i< j\leq k}}
					      \left(t_{j}-t_{i}\right)
				      }
				      =
				      \prod\limits_{\mathclap{0\leq i< j\leq k+1}}
				      \left(t_{j}-t_{i}\right).
			      \end{align*}
			      En $\left(\star\right)$, hemos aplicado las operaciones
			      \begin{math}
				      \forall j\in\left\{2,\dotsc,k+1\right\}:
				      c_{j+1}\leftarrow
				      c_{j+1}-c_{j}\cdot t_{0}
			      \end{math}.
			      $\therefore\forall n\in\mathbb{N}$, $p\left(n\right)$ es
			      verdadero.
			      % $A_{n}$ la matriz de Vandermonde asociada al sistema
			      % que halla los coeficientes del polinomio de interpolación
			      % para $n+1$ puntos distintos.
		\end{itemize}
	\end{solution}
\end{frame}