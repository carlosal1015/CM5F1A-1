\section{Pregunta N$^{\circ}$11\qquad Andre Gilmer Santos Felix}

\begin{frame}
	\begin{theorem}[El Principio de Inducción Matemática]
		Sea $F$ un \alert{cuerpo ordenado}.
		Suponga que $\forall n\in\mathbb{N}_{F}$, $p\left(n\right)$ es
		una proposición acerca de $n$.
		Si

		\begin{multicols}{2}
			\begin{enumerate}[(1)]
				\item\label{hyp:1}

				$p\left(1\right)$ es verdadero, y

				\item\label{hyp:2}

				$\forall k\in\mathbb{N}_{F}$,
				$p\left(k\right)\implies p\left(k+1\right)$,
			\end{enumerate}
		\end{multicols}

		entonces $\forall n\in\mathbb{N}_{F}$, $p\left(n\right)$ es
		verdadero.
	\end{theorem}

	\begin{proof}
		Suponga que $p\left(n\right)$ es como se describe en la
		hipótesis.
		Sea
		\begin{math}
			A=
			\left\{
			x\in\mathbb{N}_{F}:
			p\left(x\right)\text{ es verdadero}
			\right\}
		\end{math}.
		Entonces,
		\begin{enumerate}[(i)]
			\item

			      $1\in A$, por~\eqref{hyp:1}.

			\item

			      Suponga que $x\in A$. Entonces, $x\in\mathbb{N}_{F}$ y
			      $p\left(x\right)$ es verdadero.
			      Así, por~\eqref{hyp:2}, $p\left(x+1\right)$ es verdadero.
			      Esto es, $x+1\in A$.
			      Por lo tanto, $x\in A\implies x+1\in A$.
			      Finalmente, $A$ es conjunto inductivo y
			      $\mathbb{N}_{F}\subset A$.
			      Esto es, $\forall n\in\mathbb{N}_{F}$, $p\left(n\right)$
			      es verdadero.
		\end{enumerate}
	\end{proof}
\end{frame}

\begin{frame}
	\begin{enumerate}\setcounter{enumi}{10}
		\item

		      Pruebe que el
		      \begin{math}
			      \begin{vmatrix}
				      A_{n}
			      \end{vmatrix}=
			      \prod\limits_{\mathclap{0\leq i< j\leq n}}
			      \left(t_{j}-t_{i}\right)
		      \end{math},
		      donde $A_{n}$ es la matriz de Vandermonde.
	\end{enumerate}

	\begin{solution}
		Sea $n\in\mathbb{N}$ y $A_{n}$ la matriz asociada al sistema
		que halla los coeficientes del polinomio de interpolación
		para $n+1$ puntos.

		\begin{equation*}
			A^{t}_{n}\coloneqq
			\begin{bmatrix}
				1         & 1         & \cdots & 1             & 1           \\
				t_0       & t_1       & \cdots & t_{n-1}       & t_{n}       \\
				\vdots    & \vdots    & \ddots & \vdots        & \vdots      \\
				t_0^{n-1} & t_1^{n-1} & \cdots & t_{n-1}^{n-1} & t_{n}^{n-1} \\
				t_0^{n}   & t_1^{n}   & \cdots & t_{n-1}^{n}   & t_{n}^{n}
			\end{bmatrix}.
		\end{equation*}

		Sea
		\begin{math}
			p\left(n\right)\coloneqq
			\left|A_{n}\right|=
			\prod\limits_{\mathclap{0\leq i< j\leq n}}
			\left(t_{j}-t_{i}\right)
		\end{math}.
		Por el \alert{Principio de Inducción Matemática} sobre $n$,

		\begin{itemize}
			\item

			      \begin{math}
				      p\left(1\right)=
				      \begin{vmatrix}
					      A_{1}
				      \end{vmatrix}=
				      \begin{vmatrix}
					      1 & t_{0} \\
					      1 & t_{1}
				      \end{vmatrix}=
				      t_{1}-t_{0}=
				      \prod\limits_{\mathclap{0\leq i< j\leq 1}}
				      t_{j}-t_{i}
			      \end{math}
			      es verdadero.

			\item

			      Supongamos que $p\left(n\right)$ es cierto hasta un
			      $n_{0}\in\mathbb{N}$. Es decir,

			      \begin{equation*}
				      \begin{vmatrix}
					      1           & 1           & \cdots & 1           & 1             \\
					      t_{0}       & t_{1}       & \cdots & t_{n}       & t_{n+1}       \\
					      \vdots      & \vdots      & \ddots & \vdots      & \vdots        \\
					      t_{0}^{n}   & t_{1}^{n}   & \cdots & t_{n}^{n}   & t_{n+1}^{n}   \\
					      t_{0}^{n+1} & t_{1}^{n+1} & \cdots & t_{n}^{n+1} & t_{n+1}^{n+1}
				      \end{vmatrix}=
				      \prod\limits_{\mathclap{0\leq i< j\leq n_{0}}}
				      \left(t_{j}-t_{i}\right).
			      \end{equation*}
		\end{itemize}
	\end{solution}
\end{frame}

\begin{frame}
	Aplicando operaciones fila $F_{i,i-1}(-a_{0})$, $2\leq i\leq n+2$
	De ello obtenemos la siguiente matriz:
	$$\begin{pmatrix} 1&1&\cdots & 1 &1\\   0&a_1-a_0&\cdots &a_n-a_0&a_{n+1}-a_0\\   \vdots&\vdots&\ddots&\vdots&\vdots\\   0&a_1^{n+1}-a_0a_{1}^{n}&\cdots&a_{n}^{n+1}-a_0a_{n+1}^{n}& a_{n+1}^{n+1}-a_0a_{n+1}^{n}\end{pmatrix}$$
	Extrayendo $\prod_{j=1}^{n+1}(a_j-a_0)$ se obtiene:

	$$=\prod_{j=1}^{n+1}(a_j-a_0) \det\begin{pmatrix} 1& 1&\cdots&1\\a_2&a_3&\cdots&a_{n+1}\\ \vdots&\vdots&\ddots&\vdots\\a_2^{n-1}&a_3^{n-1}&\cdots&a_{n+1}^{n-1}\end{pmatrix}$$
	Luego, $p\left(n+1\right)$ se cumple.
	Por lo tanto, $p\left(n\right)$ se cumple para todo $n$ natural.
\end{frame}