\section{Pregunta N$^{\circ}$11\qquad León Alonzo Terrones Caccha}

\begin{frame}
	\begin{theorem}[El Principio de Inducción Matemática]
		Sea $F$ un \alert{cuerpo ordenado}.
		Suponga que $\forall n\in\mathbb{N}_{F}$, $p\left(n\right)$ es
		una proposición acerca de $n$.
		Si

		\begin{multicols}{2}
			\begin{enumerate}[(1)]
				\item\label{hyp:1}

				$p\left(1\right)$ es verdadero, y

				\item\label{hyp:2}

				$\forall k\in\mathbb{N}_{F}$,
				$p\left(k\right)\implies p\left(k+1\right)$,
			\end{enumerate}
		\end{multicols}

		entonces $\forall n\in\mathbb{N}_{F}$, $p\left(n\right)$ es
		verdadero.
	\end{theorem}

	\begin{proof}
		Suponga que $p\left(n\right)$ es como se describe en la
		hipótesis.
		Sea
		\begin{math}
			A=
			\left\{
			x\in\mathbb{N}_{F}:
			p\left(x\right)\text{ es verdadero}
			\right\}
		\end{math}.
		Entonces,
		\begin{enumerate}[(i)]
			\item

			      $1\in A$, por~\eqref{hyp:1}.

			\item

			      Suponga que $x\in A$. Entonces, $x\in\mathbb{N}_{F}$ y
			      $p\left(x\right)$ es verdadero.
			      Así, por~\eqref{hyp:2}, $p\left(x+1\right)$ es verdadero.
			      Esto es, $x+1\in A$.
			      Por lo tanto, $x\in A\implies x+1\in A$.
			      Finalmente, $A$ es conjunto inductivo y
			      $\mathbb{N}_{F}\subset A$.
			      Esto es, $\forall n\in\mathbb{N}_{F}$, $p\left(n\right)$
			      es verdadero.
		\end{enumerate}
	\end{proof}
\end{frame}

\begin{frame}
	\begin{enumerate}\setcounter{enumi}{10}
		\item

		      Sea $n\in \mathbb{N}$ y $A_n$ la matriz asociada al sistema que halla los coeficientes del polinomio de interpolación para $n+1$ puntos. Pruebe que para $A^{t}_{n}=\begin{pmatrix} 1&1&\cdots & 1 &1\\a_0&a_1&\cdots &a_{n-1}&a_{n}\\\vdots&\vdots&\ddots&\vdots&\vdots\\a_0^{n-1}&a_1^{n-1}&\cdots&a_{n-1}^{n-1}& a_{n}^{n-1}\\a_0^{n}&a_1^{n}&\cdots&a_{n-1}^{n}&a_{n}^{n}\end{pmatrix}$ la matriz de Vandermonde se
		      tiene
		      \begin{math}
			      \det\left(A_{n}\right)=
			      \prod\limits_{\substack{0\leq i< j\leq n}}^{n}
			      \left(a_{j}-a_{i}\right)
		      \end{math}.
	\end{enumerate}

	\begin{solution}

		Definimos la proposición
		\begin{math}
			p\left(n\right)\coloneqq
			\det\left(A_{n}\right)=
			\prod\limits_{i=0}^{j}
			\left(t_{i}-t_{j}\right)
		\end{math}.

		Por el \alert{Principio de Inducción Matemática}, vemos que
		el caso base $p\left(1\right)$(cuando hay solamente 2 puntos por interpolar) se cumple:
	\begin{equation*}
	    \det\begin{pmatrix} 
            1&a_0\\
            1&a_1\end{pmatrix}=a_1-a_0
	\end{equation*}
  %\begin{equation*}
	%		\prod\limits_{\substack{i=0\\i\neq j}}^{1}
	%		\left(x_{i}-x_{j}\right)
	%		=\det\left(A_{1}\right)
 %
		%\end{equation*}

		Para la hipótesis inductiva, asumamos que $p\left(n\right)$ se
		cumple hasta un $n\in\mathbb{N}$ fijo.
		%\begin{align*}
	%		a_{k+1} & \geq a_{k}
		%\end{align*}
        Entonces la matriz $$A^{t}_{n+1}=\begin{pmatrix} 1&1&\cdots & 1 &1\\a_0&a_1&\cdots &a_{n}&a_{n+1}\\\vdots&\vdots&\ddots&\vdots&\vdots\\a_0^{n}&a_1^{n}&\cdots&a_{n}^{n}& a_{n+1}^{n}\\a_0^{n+1}&a_1^{n+1}&\cdots&a_{n}^{n+1}&a_{n+1}^{n+1}\end{pmatrix}$$
        \end{solution}
    \end{frame}
    \begin{frame}
    Aplicando operaciones fila $F_{i,i-1}(-a_{0})$, $2\leq i\leq n+2$
     De ello obtenemos la siguiente matriz:
     $$\begin{pmatrix} 1&1&\cdots & 1 &1\\   0&a_1-a_0&\cdots &a_n-a_0&a_{n+1}-a_0\\   \vdots&\vdots&\ddots&\vdots&\vdots\\   0&a_1^{n+1}-a_0a_{1}^{n}&\cdots&a_{n}^{n+1}-a_0a_{n+1}^{n}& a_{n+1}^{n+1}-a_0a_{n+1}^{n}\end{pmatrix}$$
    Extrayendo $\prod_{j=1}^{n+1}(a_j-a_0)$ se obtiene:
    
    $$=\prod_{j=1}^{n+1}(a_j-a_0) \det\begin{pmatrix} 1& 1&\cdots&1\\a_2&a_3&\cdots&a_{n+1}\\ \vdots&\vdots&\ddots&\vdots\\a_2^{n-1}&a_3^{n-1}&\cdots&a_{n+1}^{n-1}\end{pmatrix}$$
		Luego, $p\left(n+1\right)$ se cumple.
		Por lo tanto, $p\left(n\right)$ se cumple para todo $n$ natural.
	
\end{frame}
