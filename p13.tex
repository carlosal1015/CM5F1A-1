\section{Pregunta N$^{\circ}$13\qquad Leon Alonzo Terrones Caccha}

\begin{frame}
    \begin{enumerate}\setcounter{enumi}{12}
        \item
            Demostrar que las funciones de Berstein para $n=3$ son l.i.
             
    \end{enumerate}

    \begin{solution}

    Sea la combinación lineal con $c_k\in\mathbb{R}:$
    \[\sum_{k=0}^{3}=c_{k}B_{k,3}(t)\]
    Por demostrar que $c_k=0$, $i\in[0.3]$
    \begin{align*}
        \sum_{k=0}^{3}=c_{k}B_{k,3}(t)&=\sum_{k=0}^{3}c_k\binom{3}{k}t^k(1-t)^{3-k}\\
        &=\sum_{k=0}^{3}{c_k\binom{3}{k}t^k\sum_{j=0}^{n-k}\binom{3-k}{j}{(-t)^{3-k-j}}}\\
        &=\sum_{j=0}^{3}{t^{n-j}\sum_{k=0}^{3-j}{c_k\binom{3}{k}\binom{3-k}{j}(-1)^{3-k-j}}}
    \end{align*}

    El cual es un polinomio de grado 3. Sean sus coeficientes $a_i$.
Para $j=0$, el coeficiente de $t^3$ es:
\begin{align*}
    a_{3}&=\sum_{k=0}^{3}{c_k\binom{3}{k}\binom{3-k}{0}(-1)^{3-k}}\\
    &=-c_0+3c_1-3c_2+c_3\\
    a_{2}&=\sum_{k=0}^{2}{c_k\binom{3}{k}\binom{3-k}{1}(-1)^{2-k}}\\
    &=3c_0-6c_1+3c_2\\
    a_{1}&=\sum_{k=0}^{2}{c_k\binom{3}{k}\binom{3-k}{1}(-1)^{2-k}}\\
    &=-3c_0+3c_1\\
    a_{0}&=\sum_{k=0}^{2}{c_k\binom{3}{k}\binom{3-k}{1}(-1)^{2-k}}\\
    &=c_0\\
\end{align*}


Al final se obtiene un sistema lineal en funcion del vector $c=(c_0,c_1,c_2,c_3)$. Debido a que $\{1,t,t^2,t^3\}$ es una base para los polinomios de grado 3 se tiene que $a_k=0$. Finalmente aplicando sustitución progresiva se obtiene $c_k=0$, $k\in[0.3]$.





    
    \begin{comment}
        \[
\begin{array}{cccccc}
x_0=0 & y_0=0 \\
    &     & 1 \\
x_1=1 & y_1=1 &             & \frac{\sqrt{3}-3}{6}\\
    &     & \frac{\sqrt{3}-1}{2}\\
x_2=3 & y_2=\sqrt{3}
\end{array}
\]
    \end{comment}

        
    \end{solution}
\end{frame}
\begin{comment} Para la función
              \begin{math}
                  f\left(t\right)=
                  \sqrt{t}
              \end{math}
              use diferencias divididas para construir el polinomio
              de Newton de grado $2$ para los puntos $t_{0}=0$,
              $t_{1}=1$ y $t_{2}=3$.
              \end{comment}